% Aquí comienza el préambulo
\documentclass[letterpaper, 12pt]{article}

% Soporte para acentos.
\usepackage[utf8]{inputenc}

%Español, división de sílabas
\usepackage[spanish,mexico]{babel}
%Numbers set
\usepackage{amssymb}
%Usar codificación T1
\usepackage[T1]{fontenc}

% Una forma de establecer los márgenes
\topmargin = -2cm         % default 1 cm
\oddsidemargin= 0cm       % default 2 cm
\textheight = 23cm        % default 15 cm
\textwidth = 17cm         % default 12 cm

% Paquetes que usaremos.
\usepackage{soul}
\usepackage{enumerate}
\usepackage{longtable} % Tablas grandes
\usepackage{pdflscape} % Cambiar orientación de una parte del documento
\usepackage{amsmath}
\usepackage{mathtools}
\usepackage{tikz}
\usepgflibrary{arrows}
\usepackage{listings}

\newtheorem{definicion}{Definición}
\newtheorem{nota}{Nota}
\newtheorem{teorema}{Teorema}
\newtheorem{lema}{Lema}
\newtheorem{observacion}{Observación}
\newtheorem{demostracion}{Demostración}
\newtheorem{corolario}{Corolario}
\newtheorem{ejemplo}{Ejemplo}
\newtheorem{ejemploH}{Ejemplo en Haskell}

\lstdefinestyle{customc}{
  belowcaptionskip=1\baselineskip,
  breaklines=true,
  frame=L,
  xleftmargin=\parindent,
  language=C,
  showstringspaces=false,
  basicstyle=\footnotesize\ttfamily,
  keywordstyle=\bfseries\color{green!40!black},
  commentstyle=\itshape\color{purple!40!black},
  identifierstyle=\color{blue},
  stringstyle=\color{orange},
}

% Titulo.
\title{Lenguajes de Programación\\Facultad de ciencias\\ Universidad Nacional Autónoma de México\\ Notas para el segundo examen parcial}
% Nombre de los integrantes del equipo
\author{Montoya Montes Pedro}
\begin{document}
\maketitle

\tableofcontents
\newpage

\section{Definiciones dadas por Ricardo (No formales)}
	\begin{definicion}
		Tipo: Abstracción de un conjunto de datos.
	\end{definicion}
	\begin{definicion}
		Tipo básico: Son las primitivas del lenguaje.
	\end{definicion}	
	\begin{definicion}
		Tipo abstacto: Son aquellas construidas a partir de las primitivas.
	\end{definicion}
	\begin{definicion}
		Juicio de tipo: Es un conjunto de reglas que determina la correctud de tipos de una expresión.
	\end{definicion}	
	\begin{definicion}
		Inferencia: Determina los valores de los tipos de una expresión.
	\end{definicion}
	\begin{definicion}
		Lenguaje sólido: Son aquellos que se sabe el comportamiento de sus operaciones
	\end{definicion}
	\begin{definicion}
		Lenguaje seguro: Son aquellos que las operaciones se aplican con el tipo correcto.
	\end{definicion}	
	\begin{definicion}
		Lenguaje envolvente: Es aquel que alberga código ajeno al lenguaje.
	\end{definicion}
	\begin{definicion}
		Lenguaje incrustado: Es el código escrito dentro de otro lenguaje.
	\end{definicion}
	\begin{definicion}
		Variable de tipo: Determina el tipo de cosas polimorficas.
	\end{definicion}
	\begin{definicion}
		Estado: 	
	\end{definicion}	
\newpage
\section{Ejercicio de Memoization}
\textbf{El siguiente es un ejemplo de memoization con la función "lucas".}

	\begin{lstlisting}
(define lucas (lambda (n))
  (cond
    [(= n 0) 1]
    [(= n 1) 2]
    [else (+(lucas (- n 1))(lucas (- n 2)))]))
	\end{lstlisting}
\textbf{Creacion del diccionario:}

	\begin{lstlisting}	  
(define dic (make_hash '()))
 (hash_set dic 0 1)		//Caso base 1 de lucas.
 (hash_set dic 1 2))		//Caso base 2 de lucas.  
	\end{lstlisting}	
\textbf{Creación de la memoization:}

	\begin{lstlisting}	  
(define (lucas n)	//Conservamos el nombre original. 
  (cond
    [(hash_ref dic n #f) (hash_ref dic n #f)]
    [else (let m (+(lucas (- n 1))(lucas (- n 2)))
    (begin (hash_set dic n m)m))]))	
	\end{lstlisting}	
\section{Recolector de basura}
\subsection{Tipos de recolectores de basura}
\textbf{1.Marcado y compactación.}

\begin{minipage}{1.5in}
\begin{tabular}{ccc}
\hline
\multicolumn{3}{|c|}{0x05} \\ \hline
\multicolumn{3}{|l|}{\textst{0x04}} \\ \hline
\multicolumn{3}{|l|}{\textst{0x03}} \\ \hline
\multicolumn{3}{|l|}{0x02} \\ \hline
\multicolumn{3}{|l|}{\textst{0x01}} \\ \hline
 \end{tabular}
\end{minipage}
\begin{minipage}{1.5in}
 \begin{tabular}{ccc}
\hline
\multicolumn{3}{|c|}{} \\ \hline
\multicolumn{3}{|l|}{} \\ \hline
\multicolumn{3}{|l|}{} \\ \hline
\multicolumn{3}{|l|}{0x05} \\ \hline
\multicolumn{3}{|l|}{0x02} \\ \hline
 \end{tabular}    
\end{minipage}

\textbf{2.Marcado y barrido.}

\begin{minipage}{1.5in}
\begin{tabular}{ccc}
\hline
\multicolumn{3}{|c|}{\textst{0x05}} \\ \hline
\multicolumn{3}{|l|}{0x04} \\ \hline
\multicolumn{3}{|l|}{\textst{0x03}} \\ \hline
\multicolumn{3}{|l|}{0x02} \\ \hline
\multicolumn{3}{|l|}{\textst{0x01}} \\ \hline
 \end{tabular}
\end{minipage}
\begin{minipage}{1.5in}
 \begin{tabular}{ccc}
\hline
\multicolumn{3}{|c|}{} \\ \hline
\multicolumn{3}{|l|}{0x04} \\ \hline
\multicolumn{3}{|l|}{} \\ \hline
\multicolumn{3}{|l|}{0x02} \\ \hline
\multicolumn{3}{|l|}{} \\ \hline
 \end{tabular}    
\end{minipage}

\textbf{3.Copia y pega usando semiespacios.}

\begin{minipage}{1.5in}
\begin{tabular}{ccc}
\hline
\multicolumn{3}{|c|}{0x05} \\ \hline
\multicolumn{3}{|l|}{0x04} \\ \hline
\multicolumn{3}{|l|}{\textst{0x03}} \\ \hline
\multicolumn{3}{|l|}{0x02} \\ \hline
\multicolumn{3}{|l|}{\textst{0x01}} \\ \hline
 \end{tabular}
\end{minipage}
\begin{minipage}{1.5in}
\begin{tabular}{|l|l|l|}
\hline
 0x4&  &  \\ \hline
 &  0x3&  \\ \hline
 &  &  0x2\\ \hline
0x5 &0x1  &  \\ \hline
 &  &  \\ \hline 
\end{tabular}
Heap
\end{minipage}
\begin{minipage}{1.5in}
 \begin{tabular}{ccc}
\hline
\multicolumn{3}{|c|}{} \\ \hline
\multicolumn{3}{|l|}{} \\ \hline
\multicolumn{3}{|l|}{0x05} \\ \hline
\multicolumn{3}{|l|}{0x04} \\ \hline
\multicolumn{3}{|l|}{0x02} \\ \hline
 \end{tabular}    
\end{minipage}
\begin{minipage}{1.5in}
\begin{tabular}{|l|l|l|}
\hline
 &  &  \\ \hline
 &  &  \\ \hline
 &  &  \\ \hline
 &  &  \\ \hline
0x5 & 0x4 &  0x2\\ \hline 
\end{tabular}
Heap  
\end{minipage}
\subsection{Pros y contras de cada tipo de recolector de basura}
\begin{enumerate}
	\item Marcado y compactación:\\
	-No hay fragmentación de memoria.\\
	-Eficiente en memoria.
	\item Marcado y barrido:\\
	-Deja fragmetación en memoria.\\
	-Eficiente en tiempo.
	\item Copia y pega usando semiespacios:\\
	-Pierde la mitad del heap.\\
	-Pierde tiempo en acomodo,pero no tanto como en marcado y compactación.
\end{enumerate}

\subsection{Forma de recolección}
	\begin{definicion}
		Conjunto raíz: Es aquello que se genera cuando se hace ejecución en un programa.
	\end{definicion}
	\begin{definicion}
		Formas de detección de basura:\\
		1.Con caminos.\\
		2.Con indices.
	\end{definicion}

\subsection{Tecnicas de recolección}
	Existen dos tecnicas de recolección:
	\begin{enumerate}
		\item Tiempo: Es aquel que en un periodo de tiempo se ejecuta el recolector de basura(se puede también diseñar de tal manera en la que se ejecute en intervalos de tiempo).
		\item Espacio: Es aquel en la que cada cantidad de objetos basura se ejecuta el recolector de basura.
	\end{enumerate}

\section{RP3}
\begin{enumerate}
	\item Definiciones:
	\begin{enumerate}
		\item Recolector de Basura: Es un algoritmo el cual limpia la memoria de forma automatizada.
		\item Conjunto raíz: Es aquello que se genera cuando se hace ejecución en un programa.
	\end{enumerate}
\end{enumerate}
\end{document} 

